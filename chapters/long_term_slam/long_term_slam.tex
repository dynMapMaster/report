%!TEX root = ../../report.tex
\chapter{Long-term Mapping of Dynamic Environments}
\label{long_term_mapping}

In order to maintain consistency between the map and the environment it is necessary to continuously update the map. Traditionally a map is build once but maintaining a map of a dynamic environment requires that the mapping process continues throughout the operation. This means a interaction with the localization and navigation. 

\\\\
A dynamic environment can be chaotic with changing levels of dynamics or more ordered with dynamics consistent over time or within areas. In this project the focus in on the industrial environment.
This chapter will discuss navigation and localization in a dynamic environment. An investigation into the characteristics of an industrial environment is also conducted in order to clarify the relevant properties for a dynamic mapping. Using existing SLAM algorithm for the long-term mapping is also discussed. 


For a mobile robot to be able to function in an uncontrolled environment alongside other machines and humans, it is necessary for the robot to be able to sense and avoid the highly dynamic objects. All mobile robots navigating in such an area will have a system to handle these changes. Slower changes, however, that does not pose an immediate problem for the navigation of the environment are often disregarded. This is seen in the fact that often, a the map of the environment is only generated once and then used for an extended period of time. 

The assumption is that the surroundings does not change apart from the highly dynamical objects that should not be included in the map. This assumption can become increasingly flawed as time progressed due to the fact that most real world scenarios are not perfectly static. In an office, furniture might be moved around and in industrial settings, machines might be put up, or moved, or pallets stored in an area. As the discrepancy between the robot's internal representation and the actual environment increases, so does the chance for errors in localization or navigation. In order to overcome this, it is necessary for the robot to continue the process of mapping its environment as it navigates through it.

\section{Navigation in Dynamic Environments}
Mobile robot navigation is often done based on a costmap of the environment. This map represents the world as a grid of cells with associated costs. This cost is used to direct the search towards goal, along the path with the lowest cost. When the map is inconsistent with the environment, the robot might move into a dead end. This necessitates re-planning, thus wasting time. Because the map is never updated the robot might continue to plan into the same dead end, making the system inefficient and annoy users. By updating the costmap with the dynamic mapping system, it is possible to reduce the amount of these errors.
\section{Localization with static map in dynamic environments}
Mobile robot localization is traditionally done based on a static map of the environment. This map is used to match up with the current perception of the surroundings, often from LiDar sensors. The localization of the robot is then determined by the best match between perception and internal static map. A widely used system for this is the Monte Carlo localization. With Monte Carlo localization the sensor noise is assumed independ between consecutive time steps [REF-THRUN]. This is violated when dynamic obstacles are present. Since the fact that if one measurement was too short, i.e. shorter than it should be by the static map, the next measurement in the same direction a small time later is also likely to be too short. Too short measurements indicates that, given the location of the robot’s sensors and the location of obstacles in the map, the measurement was shorter than the distance between the sensor and an obstacle. This violation stems from the unmodeled dynamics in the representation of the world. The effects of unmodeled dynamic obstacles in the world can be mitigated by detecting and removing hits on these obstacles from the observations used in the localization [REFS]. Methods for handling it this way is typically meant to reduce the effect of highly dynamical object but as they do not change the internal map, they do not improve localization as the static features in the world changes. Updating the internal map in order to incorporate the changes in the world to improve the localization is one of the tasks for dynamic learner. 

\section{Characteristics in Industrial Environments}
\label{sec:characteristics_in_industrial_environments}
An investigation of a representative industrial area where mobile robots are used, shows that areas share common dynamics. 
During a visit at SCAN A/S, a high degree of structure and fixed routines were observed. 
Different work stations produce different parts of the product, and place them in areas for the next station to continue on them.
Part of the production line at SCAN A/S, where a MiR-100 robot is used, are shown in figure \ref{fig:scan-mir}. 
A machine is fixed to the floor and is almost as static as the walls in the factory.
Figure \ref{fig:scan-semi-static-obstacles} shows product parts that have been processed by one station, and are ready to be picked up by the next.
Such routines where products part are produced at rather fixed rates and positioned in confined areas results in areas with almost constant dynamics. 
Considering this, it is reasonable to attempt to learn the dynamics of these areas.
If the objects moves at a constant rate, it might be possible to predict the presence of them in the future by learning their transient behavior.
As seen in figure  \ref{fig:scan-semi-static-obstacles}, objects are only roughly placed and are not aligned carefully, which can influence mapping with a high resolution grid.

\begin{table}[htbp]
	\caption{Regions of industrial environments}
	\label{tab:regions_of_industrial_environments}
	\begin{center}
		\begin{tabular}{p{2.cm} | p{2.6cm} | p{2.6cm} | p{2.6cm} | p{2.6cm}}
			\toprule
			\textbf{Type} & \textbf{Dominated areas} & \textbf{Object types} & \textbf{In navigation} & \textbf{In localization} \\ 
			\rowcolor[gray]{0.925}
			\textit{Highly dynamic} & Central corridors & Humans, moving vehicles & Avoid these areas if convenient & Consider avoiding these areas \\
			\textit{Semi-static} & Temporary Storage areas & Pallets with parts or products & Incorporate for next planning with non-lethal & Landmarks value based on degree of dynamics \\ 
			\rowcolor[gray]{0.925}
			\textit{Static} & Production areas & Walls, heavy machinery & Incorporate in map with lethal & Very good landmarks \\			
			\bottomrule
		\end{tabular} 
	\end{center}
\end{table}

Areas can roughly be classified based on the dynamics of the obstacles that are usually contained within them, as shown in table \ref{tab:regions_of_industrial_environments}. Areas with objects that moves almost constantly like humans are categorized as \textit{highly dynamic}, whereas areas with objects like walls, that never move are categorized as \textit{static}. The remaining objects are classified as \textit{semi-static}. This contains everything from parked vehicles that moves within minutes and production parts that may remain stationary for weeks. 

\begin{figure}[htbp]
	\centering
	\begin{subfigure}[t]{0.6\textwidth}
		\includegraphics[width=1.0\textwidth]{chapters/mapping_of_dynamic_areas/figures/scan-mir}	
		\caption{MiR-100 robot navigating close to heavy machinery.}
		\label{fig:scan-mir}
	\end{subfigure}
	\begin{subfigure}[t]{0.3875\textwidth}
		\includegraphics[width=1.0\textwidth]{chapters/mapping_of_dynamic_areas/figures/scan-semi-static-obstacles}
		\caption{\textit{Semi-static} obstacles in the form of product parts.}
		\label{fig:scan-semi-static-obstacles}
	\end{subfigure}
	\caption{Examples of obstacles in the production area at SCAN A/S.}
\end{figure}

The information of these areas can be incorporated into the representation of the world to avoid having to re-plan a path and improve localization. 

While navigating, \textit{highly dynamic} objects should not be considered as obstacles, but some re-planning of the path might be unavoidable due to them. 
It is beneficial to consider \textit{semi-static} obstacles in path planning to avoid having to re-plan, but by assigning them as lethal obstacles it might be impossible to plan a path although one might exist.
The \textit{static} objects should always be present as lethal obstacles when path planning to avoid navigating along obscure routes.

For localization, \textit{highly dynamic} objects should not be included as landmarks since they are moving almost constantly. 
Since the value of the \textit{semi-static} obstacles as landmarks depends on how dynamic they are, they should be weighted in the localization algorithm accordingly. 
The \textit{static} obstacles are very good landmarks.
\section{Long-term SLAM in Dynamic Environments}
A possible solution to handling the changes in the world is continuously running a SLAM algorithm. 
These algorithms are often used to generate the initial map of the environment but continuing to run them will generate maps that are up-to-date and can encompass the changes in the world. The challenges in using a SLAM approach is associating parts of the newly generated map with previously obtained map. 
As the world might have changed between the SLAM runs any discrepancy can either be caused by an error in localization or a change in the environment. 
The problem of data association and the errors it can lead to is described in \cite{tipaldi2013lifelong}. 
Another issue with running SLAM algorithms throughout the operation time is the memory and computational requirements of these algorithms. Using a multi-hypothesis SLAM algorithm a complete map for each hypothesis is required. This can become a problem when the area of operation becomes large. 
In order for the slam algorithms to generate good scan matching results the speed of the robot is important. If the speed is too high the quality of the map generated might be lowered. It is not desirable to limit the speed of the robot when it is performing its tasks. 


\section{Summary}
This chapter investigated the challenges involved in navigating industrial environments and the possibilities provided by a continuous mapping system. 
Industrial environments tends to have area specific dynamics. This means that learning the dynamics of an area might be possible. This information could be used to improve the navigation and localization systems.
The primary challenge considered for the navigation is repeated, dead-end path planning, where the path is blocked but the robot continues to plan through it.
For the localization the feature map should be kept up-to-date with the environment in order to maintain a good localization. 
Using a SLAM algorithm to provide long-term mapping is obstructed by the heavy resource requirements and difficulty in associating maps at different times. These algorithms are designed to handle localization and mapping with no prior knowledge and thus are not necessary due to the slow changes.  
