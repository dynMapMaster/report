\section{Localization with static map in dynamic environments}
Mobile robot localization is traditionally done based on a static map of the environment. This map is used to match up with the current perception of the surroundings, often from LiDar sensors. The localization of the robot is then determined by the best match between perception and internal static map. A widely used system for this is the Monte Carlo localization. With Monte Carlo localization the sensor noise is assumed independ between consecutive time steps [REF-THRUN]. This is violated when dynamic obstacles are present. Since the fact that if one measurement was too short, i.e. shorter than it should be by the static map, the next measurement in the same direction a small time later is also likely to be too short. Too short measurements indicates that, given the location of the robot’s sensors and the location of obstacles in the map, the measurement was shorter than the distance between the sensor and an obstacle. This violation stems from the unmodeled dynamics in the representation of the world. The effects of unmodeled dynamic obstacles in the world can be mitigated by detecting and removing hits on these obstacles from the observations used in the localization [REFS]. Methods for handling it this way is typically meant to reduce the effect of highly dynamical object but as they do not change the internal map, they do not improve localization as the static features in the world changes. Updating the internal map in order to incorporate the changes in the world to improve the localization is one of the tasks for dynamic learner. 
