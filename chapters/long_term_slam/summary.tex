\section{Summary}
This chapter investigated the challenges involved in navigating industrial environments and the possibilities provided by a continuous mapping system. 
Industrial environments tends to have area specific dynamics. This means that learning the dynamics of an area might be possible. This information could be used to improve the navigation and localization systems.
The primary challenge considered for the navigation is repeated, dead-end path planning, where the path is blocked but the robot continues to plan through it.
For the localization the feature map should be kept up-to-date with the environment in order to maintain a good localization. 
Using a SLAM algorithm to provide long-term mapping is obstructed by the heavy resource requirements and difficulty in associating maps at different times. These algorithms are designed to handle localization and mapping with no prior knowledge and thus are not necessary due to the slow changes.  