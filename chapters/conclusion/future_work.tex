\section{Future Work}

The static mapper currently incorporates the various noise by reducing the update values in the map. 
A possible improvement to the weighting could be considering the entire covariance matrix instead of the individual standard deviation. 
Using the noise more proactive could also improve the mapping results and should help converging faster towards the correct result.
Seen as the localization noise is a significant influence on the mapping any improvement in the accuracy would also benefit the mapping.


The learner showed unbalanced learning of the two transition probabilities.
It is believed that the origins of these problems stems from the static mapping. 
A mapping method that more accurately incorporates the noise could help solve this challenge. 


The classification for converting the dynamics into maps for the localization was handled by a manually designed classifier.
By obtaining vast amounts of data it might be possible to improve the classification through machine learning techniques. 


To continue the development of the Dynamic mapping system it would be beneficial to evaluate the long term performance of the system. 
This should investigate the long-term effects of the feedback between the localization and the dynamic mapping.