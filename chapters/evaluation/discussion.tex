\section{Discussion}


\subsection{Ability to Learn Dynamics of environment}
The proposed PMAC methods ability to learn the dynamic behavior was investigated by letting it learn from observations of obstacles moving with a controlled Markov process.
The errors in the estimated Markov parameters when using the PMAC learner described in section \ref{sec:learning_markov_evaluation} most likely stems from unmodeled localization errors and insufficient number of observations. 

The wrong estimates for lambda $\lambda_{entry}$ might be due to the fact that cells near the edges of obstacles are often wrongly observed free when the end of the ray-traced lines are too long due to errors in the estimated robot pose.
When the lines are too short due to localization errors the cells further from the laser in the direction of the line are however not counted up as occupied.
This causes an increase in free count values compared to the ones in the entry event counter, which results in the $\lambda_{entry}$ values being too small.
The estimated $\lambda_{exit}$ do not suffers from this problem and are also better, but the small amount of measurements is simply not sufficient to achieve accurate estimates for all the boxes.

Despite the fact that it cannot be shown that PMAC learns the dynamics of semi-static obstacles fast,
the method is useful to improve prediction of future probability for occupied and location of static obstacles.

\subsection{Prediction of Obstacles Location}
When predicting five minutes out into the future in between updates a better match with the observed observation is achieved when using the proposed predict-update method, than when just predicting with the previously observed probability for occupied.
It should be noted that these results are not general, since the accuracy of the prediction depends on the time between observations and how dynamic the observed obstacle are.

\subsection{Localization with Dynamic Map}
The estimation certainty of the localization performed by AMCL is smaller when using a dynamic map.

Jumps in estimated position to an invalid pose is avoided.

The accuracy and precision of the localization is better when using a dynamic map representation.
The navigation system arrives faster at an accurate pose.


