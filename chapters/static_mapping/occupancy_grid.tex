%!TEX root = ../../report.tex
\section{Map representation}
\label{sec:occupancy_grid_mapping}

An initial decision in mapping is how to represent the environment. A widely used method is the occupancy grid map (OG), described in \cite{elfesMoravecOccGrid}. It splits the environment into evenly sized cells. 
Each cell contains a probability, which indicates how plausible it is that the area it spans over contains an obstacle.
A zero indicates that the cell is free and a one indicates that the cell is occupied.


Large occupancy grid maps take up a lot of memory as the entire environment is represented in the grid map. 
The benefit of the grid map is an ease of maintenance and clear metric information \cite{mapbuildingSummary}. 
Another widely used representation is the topological map \cite{topologyOrig}. 
Herein, the environment is represented as position or landmark nodes connected by vertices in a graph structure. 
This representation was created for effective navigation. 
A drawback of the topological map is that it contains less metric information than the occupancy grid mapping \cite{mapbuildingSummary}.

For the static mapping in this project there are other factors influencing the choice of representation. In the widely used Robot Operating System (ROS)\cite{ros_info} the grid map is used. 
Hence, by using a grid map, integrating with other ROS systems becomes easier.
The availability of metric information, widespread use and ease of integration makes the occupancy grid the most suitable map representation for this project. 

\subsection{Occupancy Grid Mapping}
When mapping with an occupancy grid each observation is assumed independent and thus adds a contribution to the occupancy of a cell. 
Another assumption is that each cell is independent, which makes the solution to the equation describing the probability for occupancy, given past measurements and positions tractable for usable sized maps \cite{Merali_Patchmap}.
The most simple mapping with occupancy grids marks a cell in accordance to its latest observation. If the cell is observed as occupied the cell value is set to one and zero if it is observed free.
This makes the model very susceptible to any form of noise. 
To avoid this, inverse sensor models that incorporate the noise of the sensor is often used.

In the occupancy grid a log-odds representation of the probability can be used \cite{probRob}. 
Equation \ref{eq:occupancy_update} descries how the existing information is updated with the inverse sensor model, the term furthest to the right. $p(m_i|z_{1:t},x_{1:t})$ indicates the probability for the cell, $m_i$ to take on the states; occupied or free, given all the past measurements $z_{1:t}$ and robot poses $x_{1:t}$. 

\begin{equation}
	log \frac{p(m_i|z_{1:t},x_{1:t})}{1-p(m_i|z_{1:t},x_{1:t})} = log \frac{p(m_i|z_{1:t-1},x_{1:t-1})}{1-p(m_i|z_{1:t-1},x_{1:t-1})} + log \frac{ p(m_i | z_t,x_t) }{ 1 - p(m_i | z_t,x_t) }
	\label{eq:occupancy_update}
\end{equation}
No prior knowledge is included because it is zero, when no time consistent information about the map is available.
As this project focuses on mapping a changing environment this consistency cannot be assumed.
It is possible to initialize the first term with initial values and thereby including an initial map and still enable new measurements to overwrite it.