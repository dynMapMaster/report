%!TEX root = ../../report.tex
\section{Occupancy grid mapping}

An initial decision in mapping is how to represent the environment. A widely used method is the occupancy grid map, described in \cite{elfesMoravecOccGrid}. It splits the environment into evenly sized cells. Each cell contains an occupancy value describing probability the cell being occupied. 0 for completely free and 1 for completely occupied. An assumption in the occupancy grid map is that each cell is independent which is not strictly true but helps reduce the complexity of the mapping [**REFERENCE INSERT HERE***]. This can be expensive in memory as the entire environment is represented in the grid map. The benefit of the grid map is an ease of maintenance and good metric information \cite{mapbuildingSummary}. 
Another widely used representation is the topological map \cite{topologyOrig}. 
Herein the environment is represented as position or landmark nodes connected by vertices in a graph structure. 
This representation was created for effective navigation in the environment. 
A drawback of the topological map is that it contains less metric information than the occupancy grid mapping \cite{mapbuildingSummary}.

For the static mapping in this project there are other factors influencing the choice of representation. In the ROS system the grid map is already used and therefore the interfacing becomes significantly simpler and thus rendering the proposed mapping more easily integrated in other systems. The availability of metric information and the widespread use and therefore ease of integration makes the occupancy grid a favored choice in this project. 

hen mapping with an occupancy grid each observation is treated as independent and thus adds a contribution to the occupancy of the cell. The most simple mapping in occupancy grid will mark a cell in accordance to its last observation with the full value. If the cell is observed as occupied the cell value is set to 1 and 0 if it is observed free. Using this very simple mapping the last observation is fully in control of the cell value. This makes the method very susceptible to any form of noise. To avoid this, values based on a sensor model are often used to compensates for the noise in the sensor. 

In the occupancy grid a log-odds representation of the probability is used \cite{probRob}. In equation \ref{eq:log-odds} the calculation from probability to log-odds is shown. The value \(p(m_{occ}|z)\) is determined by the sensor model. 

\begin{equation}
\label{eq:log-odds}
l_{occ} =  log(\frac{p(m_{occ}|z)}{1-p(m_{occ}|z)} )
\end{equation}

where \(z\) is the observation and \(m_{occ}\) is the occupied state. 
\\
Using the log-odds representation the occupancy value of a cell is calculated by equation \ref{eq:occGrid_occCalc}. 

\begin{equation}
\label{eq:occGrid_occCalc}
cell_{occ} = \sum_{i} log(\frac{p(m_{occ}|z_i)}{1-p(m_{occ}|z_i)} )
\end{equation}



Why occupancy grid ? (interface, recognised, used through put the community - considerations on other representation)
Mapping by inserting obstacle and free when observed.
Independence - cells, measurements
Log odds
Real world test -> Results in bad maps.
Propagation of noise to avoid over confidence while still converging to the right solution. (Types of noise)

