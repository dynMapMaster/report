%!TEX root = ../../report.tex
\section{Occupancy grid mapping}

An initial decision in mapping is how to represent the environment. A widely used method is the occupancy grid map, described in [REFERENCE, ELFE?]. It splits the environment into evenly sized cells. Each cell contains an occupancy value describing how occupied the cell is. Often 0 for completely free and 1 for completely occupied.  
Another widely used representation is the topological map \cite{topologyOrig}. Herein the environment is represented as position or landmark nodes connected by vertices in a graph structure. This representation was created for effective navigation in the environment. A drawback of the topological map is that it contains less metric information than the occupancy grid mapping \cite{mapbuildingSummary}.


Why occupancy grid ? (interface, recognised, used through put the community - considerations on other representation)

Mapping by inserting obstacle and free when observed.

Log odds

Real world test -> Results in bad maps.

Propagation of noise to avoid over confidence while still converging to the right solution. (Types of noise)

